\documentclass{beamer}
\usepackage[utf8]{inputenc}
\usepackage{graphicx}
\usetheme{Madrid}
\usetheme{Antibes}
\usetheme{tree}
\usetheme{classic}

%%%%%%%%%%%%%%%%%%%%%%%%%%%%%%%%%%%%%%%%%%%%%%%%%%%%%%%%%%%%%%%%%%%%%%%%%%%%%%%
\title[El Número PI]{Presentación con Beamer}
\author{Jorge Antonio Herrera Alonso}
\date[25-04-2014]{25 de abril de 2014}
%%%%%%%%%%%%%%%%%%%%%%%%%%%%%%%%%%%%%%%%%%%%%%%%%%%%%%%%%%%%%%%%%%%%%%%%%%%%%%%
\begin{document}
  
%++++++++++++++++++++++++++++++++++++++++++++++++++++++++++++++++++++++++++++++
\begin{frame}

  \begin{small}
    \begin{center}
     Presentación sobre el número PI \\
     Técnicas Experimentales \\
     2014 \\
     $\pi$ $\approx 3.14159265358979323846... $
    \end{center}
  \end{small}

\end{frame}
%++++++++++++++++++++++++++++++++++++++++++++++++++++++++++++++++++++++++++++++

%++++++++++++++++++++++++++++++++++++++++++++++++++++++++++++++++++++++++++++++
\begin{frame}
  \frametitle{Historia del Número $\pi$}
  \tableofcontents[pausesections]
\end{frame}
%++++++++++++++++++++++++++++++++++++++++++++++++++++++++++++++++++++++++++++++


\section{\bf Referencias bíblicas}


%++++++++++++++++++++++++++++++++++++++++++++++++++++++++++++++++++++++++++++++
\begin{frame}

\frametitle{\bf Referencias bíblicas}

Una de las referencias indirectas más antiguas del valor aproximado de $\pi$ se puede encontrar en un versículo de la Biblia:

 "Hizo fundir asimismo un mar de diez codos de un lado al otro, perfectamente redondo. Tenía cinco codos de altura y a su alrededor un cordón de treinta codos."


\end{frame}
%++++++++++++++++++++++++++++++++++++++++++++++++++++++++++++++++++++++++++++++

\section{\bf Antigüedad Clásica}

%++++++++++++++++++++++++++++++++++++++++++++++++++++++++++++++++++++++++++++++
\begin{frame}

\frametitle{\bf Antigüedad Clásica}

  
  El matemático griego Arquímedes (siglo III a. C.) fue capaz de determinar el valor de $\pi$ entre el intervalo comprendido por 3 10/71, como valor mínimo, y 3 1/7, como valor máximo. Con esta aproximación de Arquímedes se obtiene un valor con un error que oscila entre 0,024 y 0,040 sobre el valor real. El método usado por Arquímedes5 era muy simple y consistía en circunscribir e inscribir polígonos regulares de n-lados en circunferencias y calcular el perímetro de dichos polígonos. Arquímedes empezó con hexágonos circunscritos e inscritos, y fue doblando el número de lados hasta llegar a polígonos de 96 lados.

  Alrededor del año 20 d. C., el arquitecto e ingeniero romano Vitruvio calcula $\pi$ como el valor fraccionario 25/8 midiendo la distancia recorrida en una revolución por una rueda de diámetro conocido.
  \end{frame}
  
  \begin{frame}

  En el siglo II, Claudio Ptolomeo proporciona un valor fraccionario por aproximaciones:

  $\pi$ =  $377/120 \approx 3.1416... $
  

\end{frame}
%++++++++++++++++++++++++++++++++++++++++++++++++++++++++++++++++++++++++++++++

\section{\bf Características Matemáticas}

%++++++++++++++++++++++++++++++++++++++++++++++++++++++++++++++++++++++++++++++
\begin{frame}
\frametitle{Características:}

Euclides fue el primero en demostrar que la relación entre una circunferencia y su diámetro es una cantidad constante. No obstante, existen diversas definiciones del número $\pi$, pero las más común es:

    $\pi$ es la relación entre la longitud de una circunferencia y su diámetro.

  Por tanto, también $\pi$ es:

    El área de un círculo unitario (de radio unidad del plano euclídeo).
    El menor numero real x positivo tal que $sin(x) = 0$.
\end{frame}

\begin{frame}

  También es posible definir analíticamente $\pi$; dos definiciones son posibles:

    La ecuación sobre los números complejos \[ e^{ix}+1=0 \] admite una infinidad de soluciones reales positivas, la más pequeña de las cuales es precisamente $\pi$ (véase identidad de Euler).
    La ecuación diferencial \[ S''(x)+S(x)=0 \] con las condiciones de contorno \[ S(0)=0, S'(0)=1 \] para la que existe solución única, garantizada por el teorema de Picard-Lindelöf, es un función analítica (la función trigonométrica $sin(x)$ cuya raíz positiva más pequeña es precisamente $\pi$.

\end{frame}
%++++++++++++++++++++++++++++++++++++++++++++++++++++++++++++++++++++++++++++++

\section{Bibliografía}
%++++++++++++++++++++++++++++++++++++++++++++++++++++++++++++++++++++++++++++++
\begin{frame}

  \frametitle{Bibliografía}
  
  \begin{thebibliography}{10}
    \beamertemplatebookbibitems
    \bibitem[Historia del Número $\pi$, 2010]{plan}
    Historia del Número $\pi$.
    (2010)

    \beamertemplatebookbibitems
    \bibitem[Los Secretos del Número $\pi$, 2011]{guia}
    Los Secretos del Número $\pi$.
    (2011)

\end{thebibliography}
\end{frame}

%++++++++++++++++++++++++++++++++++++++++++++++++++++++++++++++++++++++++++++++
\end{document}